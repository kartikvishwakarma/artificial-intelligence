\documentclass[32pt]{article}
\usepackage{hyperref}
\usepackage[margin=1.10in]{geometry}

\usepackage{listings}
\usepackage{framed}
\usepackage{color}
\usepackage{tikz}
\usetikzlibrary{arrows}

\newcommand{\specialcell}[2][c]{%
  \begin{tabular}[#1]{@{}c@{}}#2\end{tabular}}
\newcommand*{\Comb}[2]{{}^{#1}C_{#2}}%

%New colors defined below
\definecolor{codegreen}{rgb}{0,0.6,0}
\definecolor{codegray}{rgb}{0.5,0.5,0.5}
\definecolor{codepurple}{rgb}{0.58,0,0.82}
\definecolor{backcolour}{rgb}{0.95,0.95,0.92}

\renewcommand{\lstlistingname}{Code}
\lstdefinestyle{python}{
  backgroundcolor=\color{backcolour},   commentstyle=\color{codegreen},
  keywordstyle=\color{magenta},
  numberstyle=\tiny\color{codegray},
  stringstyle=\color{codepurple},
  basicstyle=\footnotesize,
  breakatwhitespace=false,         
  breaklines=true,                 
  captionpos=b,                    
  keepspaces=true,                 
  numbers=left,                    
  numbersep=5pt,                  
  showspaces=false,                
  showstringspaces=false,
  showtabs=false,                  
  tabsize=2
}

 \lstset{escapeinside={(*@}{@*)}}

\title{Lab4 - Planning}
\author{Naman Goyal (2015CSB1021)}
\begin{document}

\vspace{-0.5cm}
\maketitle



\section{Planners for Block World}
\paragraph{Objective}
To compare the performance of various planners


\begin{itemize}
   \item{Forward (progression) planner using breadth first search}
\item{Forward (progression) planner using A* search}
\item{Goal Stack planner}
\end{itemize}

\subsection{Statistics}
\begin{center}
1.txt (4 blocks) \\
\begin{tabular}{ |c|c|c|c|c| } 
 \hline
 \textbf{Performance Metric} & \textbf{BFS} & \textbf{\specialcell{Admissible\\A* Search}} & \textbf{\specialcell{Inadmissible\\A* Search}} & \textbf{Goal Stack}  \\ 
 \hline]
 
  \textit{Search Time (in sec)} & 0.018 & 0.003 & 0.003 & 0.003 \\ 
 \hline
 \textit{Path Length} & 10 & 10 & 10 & 18 \\ 
  \hline
 \textit{Number of Nodes expanded} & 109 & 21 & 19 & -- \\ 
 \hline
  \textit{Size of Goal Stack} & -- & -- & -- & 9 \\ 

 \hline
\end{tabular}
\end{center}

\begin{center}
3.txt (5 blocks) \\
\begin{tabular}{ |c|c|c|c|c| } 
 \hline
 \textbf{Performance Metric} & \textbf{BFS} & \textbf{\specialcell{Admissible\\A* Search}} & \textbf{\specialcell{Inadmissible\\A* Search}} & \textbf{Goal Stack}  \\ 
\hline
 
  \textit{Search Time (in sec)} & 0.024 & 0.004 & 0.004 & 0.003 \\ 
 \hline
 \textit{Path Length} & 14 & 14 & 16 & 18 \\ 
  \hline
 \textit{Number of Nodes expanded} & 499 & 45 & 42 & -- \\
 \hline
  \textit{Size of Goal Stack} & -- & -- & -- & 15 \\ 

 \hline
\end{tabular}
\end{center}

\begin{center}
5.txt (12 blocks) \\
\begin{tabular}{ |c|c|c|c|c| } 
 \hline
 \textbf{Performance Metric} & \textbf{BFS} & \textbf{\specialcell{Admissible\\A* Search}} & \textbf{\specialcell{Inadmissible\\A* Search}} & \textbf{Goal Stack}  \\ 
\hline
 
  \textit{Search Time (in sec)} & -- & 4.012 & 0.145 & 0.004 \\ 
 \hline
 \textit{Path Length} & -- & 26 & 26 & 30 \\ 
  \hline
 \textit{Number of Nodes expanded} & -- & 22040 & 936 & -- \\
 \hline
  \textit{Size of Goal Stack} & -- & -- & -- & 19 \\ 
 \hline
\end{tabular}
\end{center}

\fbox{\parbox{\textwidth}{BFS was not used for 12 blocks case since BFS expands all nodes and hence becomes memory intensive. Also BFS takes a huge time since it parses level by level to reach goal state.}}

\subsection{Analysis}
\begin{enumerate}
    \item 
    \textbf{Observation} 
    
    A* Search reduces the search time and number of nodes expanded significantly over BFS 
    
    \textbf{Explanation} 
    
    A* Search tries expanding those nodes which are more closer to goal state using heuristic value.
    
     \item 
    
    \textbf{Observation} 
    
    Inadmissible heuristics gives plan quickly but may return suboptimal plan over the admissible one.
    
    \textbf{Explanation} 
    
    Inadmissible heuristics overestimates the value of reaching the goal state and hence may return suboptimal answer.
    
    
    \item 
    
    \textbf{Observation} 
    
    Goal Stack planning is faster than other two and requires much less memory but returns suboptimal plan.
    
    \textbf{Explanation} 
    
    Goal Stack Planning is a Goal Directed Search hence finds solution quickly; but Sussman's anomaly proves that Goal Stack Planning is suboptimal.
    
    
\end{enumerate}

\vfill

\subsection{Heuristics}

Heuristics returns an estimated value of arriving at goal state.

It returns the estimated cost of solving a relaxed version of original sub-problem. The constraint that robotic arm can hold only one block is discarded and the robotic hand can now hold any number of blocks called \textbf{pool}. A block is added to pool only once using a cost of 1 step and is then removed from the pool again.

Cost of removing a block for admissible case is 1.
While the cost of removing a block for 'inadmissible' case is 1 (if block is ontable) and 3 (if block is on another block). Each of the unsatisfied literals in goal state is considered and the cost estimates of satisfying them are added.
\vfill
Consider the situation

\begin{center}
\textbf{Current State}
(on 1 2) (on 2 3) (ontable 3) (clear 1) (empty)

\textbf{Goal State}
(on 1 2) (ontable 2) (ontable 3) (clear 3) (clear 1) (empty)
\end{center}


To satisfy (Clear 3) all the blocks above 3 i.e. 1 and 2 are put into the pool. Since none of them was already in pool the cost is 2 steps.

Then since the goal state has arm empty; 1 and 2 are removed from the pool.

For admissible case, the blocks 1 & 2 are removed from pool using a step cost 1 each i.e. total 2 steps.

For inadmissible case, the block 1 (on 2) is removed using a cost of 3 and block 2 (ontable) is removed using a cost of 1 from pooli.e. total 4 steps.

Hence Heuristic Value is 4 (for admissible) and 6 (for inadmissible) while actual cost is 6
\vfill
\newpage
Consider the situation

\begin{center}
\textbf{Current State}
(on 1 3) (on 3 4) (ontable 4) (ontable 2) (clear 1) (clear 2) (empty)

\textbf{Goal State}
(ontable 1) (ontable 2) (ontable 3) (ontable 4) (clear 1) (clear 2) (clear 3) (clear 4) (empty)
\end{center}

To satisfy (Clear 3) all the blocks above 3 i.e. 1 is put into the pool. Since 1 was  not already in pool the cost is 1 step.

Then to Satisfy (Clear 4) all the blocks above 4 i.e. 1 and 3 are put into the pool. Since 1 was already in pool the cost is 0 step while 3 was not in pool so cost is 1 step. Net cost of putting in pool is $1 + 0 = 1$ step

For admissible case, the blocks 1 & 3 are removed from pool using a step cost 1 each i.e. total 2 steps.

For admissible case, the blocks 1 & 3 are removed from pool using a step cost 1 each (as both on table) i.e. total 2 steps.

Hence Heuristic Value is 4 (for admissible) and 4 (for inadmissible) while actual cost is 4.





\end{document}
